\documentclass[letterpaper,11pt]{article}
\usepackage[parfill]{parskip} % Remove paragraph indentation
\usepackage{amsmath}
\usepackage{float}
\usepackage[margin=1in]{geometry}
\usepackage{graphicx}
\usepackage{placeins}
\usepackage{siunitx}
\usepackage[title]{appendix}
\usepackage{pdflscape}
\usepackage{tabularx}
\usepackage{times}
\usepackage{url}
\usepackage{setspace}
\usepackage[none]{hyphenat}
\usepackage{pdfpages}
\usepackage{longtable}
\usepackage{tikz}
\usepackage{hyperref}
\usepackage{minted}

\DeclareSIUnit{\samplepersec}{SPS}

\begin{document}

\begin{titlepage}
    \begin{center}
        \vspace*{1cm}

        \Large
        \textbf{ELEC 490/498 Project Final Report}

        \vspace{0.5cm}
        Group 18\\
        TeachEE: Accessible Electronics Instrumentation\\
        \vspace{0.5cm}
        \normalsize
        \textbf{John Giorshev (20103586, john.giorshev@queensu.ca) \\ Eric Yang (20120750, e.yang@queensu.ca) \\ Ethan Peterson (20105011, 17emp5@queensu.ca) \\ Timothy Morland (20096286, 17tm22@queensu.ca)}\\
        \vspace{0.5cm}
        Submitted April 10, 2023\\

        \vfill
            
        \textbf{To:}\\
        Instructor Dr. Michael Korenberg (korenber@queensu.ca) \\
        Instructor Dr. Alireza Bakhshai (alireza.bakhshai@queensu.ca) \\
        Instructor Dr. Alex Tait (alex.tait@queensu.ca) \\
        Instructor and Supervisor Dr. Sean Whitehall (sw109@queensu.ca) \\
        Supervisor Dr. Thomas Dean (tom.dean@queensu.ca) \\
            
        \vspace{1.8cm}

    \end{center}
\end{titlepage}
\setstretch{1.5}
\pagenumbering{gobble}
\section*{Executive Summary}
% ERIC OWNS THIS SECTION
EXEC SUMMARY HERE

\newpage

\setstretch{1}
\tableofcontents
\listoffigures
\listoftables
\newpage
% upgraded this to double spacing as per report formatting reqs
\setstretch{2}
\pagenumbering{arabic}
\section{Motivation \& Background}
% ERIC
% Motivation and background – a clear problem statement and the motivation for the
% project (why is it technologically interesting?). A good motivation argument
% usually relies on facts and figures about the technological void that you seek
% to fill with your design. Back-up your facts and figures by referencing archival
% references. Examples of archival references include: journal papers, conference
% papers, patents, books, corporate technical and annual reports, application
% notes. Use the standard IEEE citation format. Website URLs are not archival

\section{Design}
%% Design – describe your functional requirements and constraints; provide
%% technical details about your design process for meeting the project goals;
%% this might include identifying subsystems, analysis, modeling, and key
%% decisions made. If your project involves circuit design, you should describe
%% the simulations used to create your design

% Summarize Design process of hardware, fpga and SW. Make a brief note about how
% we parallelized much of the work by modelling the data as a byte stream.
% allowing for significant dev work without physical HW in house.

The TeachEE design is derived from the system requirements set in the blueprint
report. Given in Table \ref{tab:hw-reqs} and \ref{tab:sw-reqs} are the hardware
and software requirements respectively. It should also be noted that the FPGA
shares responsibility with the desktop application for fulfilling software
requirements.

\begin{table}[H]
    \caption{Hardware Requirements}
    \begin{tabularx}{\textwidth}{l|l|l|l}
          & Specification & Target Value & Tolerance \\
        \hline
        1 &Voltage Input Bandwidth&\SI{100}{\kilo\hertz}& $\pm \SI{1}{\kilo\hertz}$ \\
        2 &Current Input Bandwidth&\SI{100}{\kilo\hertz}& $\pm \SI{1}{\kilo\hertz}$ \\
        3 &Measureable Current Range&\SI{-15}{\ampere} to \SI{+15}{\ampere}& $\pm \SI{5}{\ampere}$ \\
        4 &Measureable Voltage Range&\SI{0}{\volt} to \SI{3.3}{\volt}& $\pm \SI{200}{\milli\volt}$ \\
        5 &Number of Current Input Channels& $1$ & $0$ \\ 
        6 &Number of Voltage Input Channels& $1$ & $+2$ \\
        7 &Power Input Voltage Rating& \SI{5}{\volt} & $\pm \SI{500}{\milli\volt}$ \\
        8 &Power Current Consumption Rating& \SI{500}{\milli\ampere} & $\pm \SI{250}{\milli\ampere}$ \\
        9 &Voltage Sample Rate& \SI{1}{\mega\samplepersec} & \SI{0}{\mega\samplepersec}\\
        10 &Current Sample Rate& \SI{1}{\mega\samplepersec} & \SI{0}{\mega\samplepersec} \\
        11 &PCB Thickness& \SI{1.6}{\milli\metre} & $\pm \SI{0.1}{\milli\metre}$ \\
        12 &PCB Dimensions& \SI{0.04}{\meter\squared} & $\pm \SI{400}{\milli\metre\squared}$ \\
        13 &Voltage Sample Error against commercial scope & N/A & $\pm \SI{20}{\percent}$ \\
        14 &Current Sample Error against commercial meter& N/A & $\pm \SI{20}{\percent}$
    \end{tabularx} 
\label{tab:hw-reqs}
\end{table}

\begin{table}[H]
  \caption{Software Requirements}
  \begin{tabularx}{\textwidth}{l|l}
    \textbf{1} & \textbf{Functional Requirements}\\
    \hline
    1.1 & The software shall be able to modify the horizontal and vertical scales of the plot. \\
    1.2 & The software shall be able to modify the trigger voltage. \\
    1.3 & The application shall be deployable to Windows, macOS, and Linux. \\
    \hline
    \textbf{2} & \textbf{Interface Requirements} \\
    \hline
    2.1 & The software shall be able to capture voltage samples and export them to a CSV file. \\
    2.2 & The software shall receive samples via the FTDI 232 in synchronous mode. \\
    \hline
    \textbf{3} & \textbf{Performance Requirements} \\
    \hline
    3.1 & The software shall be able to render waveforms at a rate of \SI{30}{\hertz} on screen.
  \end{tabularx} 
  \label{tab:sw-reqs}
\end{table}

Based on the system requirements, the project is broken up into three
subsystems; hardware, FPGA, and software. The following subsections correspond
to each subsystem. Each section describes the subsystem's top level design, and
key design decisions made to comply with system requirements.

% INCLUDE THE BLOCK DIAGRAMS FOR EACH OF THESE
% Talk about old design (from blueprint)
\subsection{Hardware} % Ethan
The primary design driving requirements for the Printed Circuit Board (PCB) are
the bandwidth and sample rate of the data acquisition channels. Bandwidth and
sample rate constrain the selection of the input filters and Analog to Digital
Converters (ADCs) on the PCB. After selecting these parts, they can be easily
duplicated to add additional channels. The power requirements for the hardware
are trivial in comparison as they are already in alignment with the power
interface supplied by a standard USB port. The physical dimension requirements
of the PCB are also comparatively simple as the parts are easily packed more
closely using modern PCB layout and assembly techniques. A full requirement
compliance check for the hardware and other subsystems is provided in Section
\ref{sec:testing}.

Given the requirement for a single voltage and current channel with room for
additional voltage channels, the team settled on a four-channel design. One ADC
channel would be used for current and another for voltage. The remaining two
channels would be made available to the on-board FPGA should the team have the
time to add additional channels. The two spare channels would also have a higher
sample rate to improve the utility of the instrument.

It should be noted that all the data acquisition channels are connected to an
FPGA rather than a microprocessor. This decision was made to ensure we had no
real time data processing limitations and precise timing in our sampling of the
signal waveforms. Moreover, it is common with existing commercial oscilloscopes
to have an FPGA or ASIC fulfill these real time requirements. One compromise
that was made in the design was the use of an FPGA module rather than installing
the chip to the PCB directly. While such a module could not be used in TeachEE
in a mass-production setting, it greatly shortened the PCB design time, and
allows the FPGA to be easily replaced in the event of hardware failure.
Moreover, the FPGA combines all the sample streams and provides a single point
of contact to the USB interface.

Figure \ref{fig:hw-block-diagram} is a complete hardware block diagram used as a
guide for the PCB design implementation.

\begin{figure}[h]
  \centering
  \includegraphics[width=\textwidth]{../../misc/TeachEE-System-Diagram-Hardware.png}
  \caption{Hardware System Block Diagram}
  \label{fig:hw-block-diagram}
\end{figure}

\subsection{FPGA} % Ethan
The FPGA RTL is primarily driven by the implicit requirement to send multiple
streams of data with low latency. The sample rate and bandwidth requirements are
not as concerning in the design of the FPGA subsystem. This is because sample
rates can be easily tuned in the FPGA with clock generation PLLs and bandwidth
depends on the analog bandwidth of the input circuitry on the PCB.

In order to transmit multiple streams, the FPGA needs to frame multi-channel
data in a standard packet format. A multiplexer is also used for a case where
sending all data at once is too much throughput for USB 2.0. With these issues
in mind, the diagram given in Figure \ref{fig:fpga-block-diagram} provides a
system-level block diagram for the FPGA design.

\begin{figure}[h]
  \centering
  \includegraphics[width=\textwidth]{../../misc/TeachEE-System-Diagram-FPGA.png}
  \caption{FPGA System Block Diagram}
  \label{fig:fpga-block-diagram}
\end{figure}

A full breakdown of Figure \ref{fig:fpga-block-diagram} is given in the Section
\ref{sec:impl} focusing on implementation. Overall, the initial design is nearly
identical to the final implementation. The team made use of COBs packet encoding
as promised in the blueprint and added a standard AXI Streaming protocol for the
internal FPGA datapath.

\subsection{Software} % Eric
% Tauri, two threaded one buffer guarded with mutex
\section{Implementation} \label{sec:impl}
%% Describe how the solution was implemented – this may involve a description of
%% major code blocks, schematics, photographs, CAD drawings, etc. The reader
%% should understand the materials and operation of your implemented project,
%% and the tools (hardware and/or software) used. This should also include a
%% full bill of materials and final project budget.

% Discuss how everything was implemented here
% Ethan will include figure of top level altium schematic
% Ethan will also include snippet of the top level SystemVerilog Module.
% I will pull in full schematics, layout and code into the appendices

The implementation is broken up into three subsections corresponding to each
subsystem.

\subsection{Hardware} % ETHAN
The hardware solution for the TeachEE is implemented in the form of a four-layer
PCB. The PCB is designed using the EDA CAD software Altium Designer. Altium is
the industry standard tool for PCB design. Our design takes advantage of
Altium's hierarchical schematics. This allows the schematic to be broken up into
multiple pages with interconnects between the different sub-circuits. The
TeachEE's top level schematic is an interconnect between all the sub-circuits of
the PCB. The top level schematic is shown in Figure
\ref{fig:hw-top-level-sheet}. It should be noted that all 8 pages of the
schematic are provided in Appendix \ref{appendix:schematic}.

\begin{figure}[h]
  \centering
  \includegraphics[width=\textwidth]{figures/altium-top-level.png}
  \caption{TeachEE PCB Top Level Schematic}
  \label{fig:hw-top-level-sheet}
\end{figure}

The following subsections break down each sub-circuit sheet entry in Figure
\ref{fig:hw-top-level-sheet}.

\subsubsection{Power}
The power schematic sheet contains a single LD1117V33C linear voltage regulator.
The regulator takes the 5V VBUS supply from the USB connection and provides a
3.3V power output. A linear voltage regulator is employed over a switching buck
converter for the following reasons.
  \begin{enumerate}
    \item The analog to digital converters require clean voltage references to
      produce accurate readings and function correctly. Buck converters are
      typically have more noisy outputs that could harm performance.
    \item The linear voltage regulator requires less components to implement
      easing chip shortage concerns.
    \item The significantly lower efficiency of a linear regulator is not an
      issue as this regulator has a capacity of 1A which is more than enough to
      run the devices on-board.
  \end{enumerate}
\subsubsection{USB FIFO}
The USB FIFO is a queue that sends data out to the laptop over USB 2.0 Full
Speed (FS). USB 2.0 FS is capable of up to 480 Mpbs. This circuit takes a
parallel byte stream from the FPGA and converts it to a USB 2.0 data stream. The
FT232HQ USB FIFO made by FTDI is used to implement this functionality. This chip
was selected over other USB Interface ICs for the following reasons.
  \begin{enumerate}
    \item Availability. USB Interfaces are one of many ICs in short supply
      in the ongoing chip shortage. In fact, this IC was ordered for the
      project after ELEC390 in July 2021 and only received in August 2022.
    \item Extensive pre-existing software tooling. FTDI provides a driver and
      configuration utility for extracting data from their USB interfaces. This
      significantly reduces the software development work required to build a
      sample streaming interface. This ease of use is critical in a project with
      such a short timeline.
  \end{enumerate}

For debugging and circuit visibility, all digital control signals on the
FT232HQ are connected to testpoints. The schematic sheet also contains some
surrounding circuitry in the form of power filtering, a clock oscillator, and
configuration EEPROM.

\subsubsection{Current Monitor}
The current monitor page contains a hall effect current sensor circuit. The
circuit makes use of the ACS720KLATR current sensor. This sensor was chosen due
to its supply chain availability and the fact that it reports its reading
through a voltage output. The voltage output is linearly proportional to the
current flowing through the sensor plus a DC offset. This is a preferred
interface as it can be simply connected to a ADC channel without having to write
a whole new driver for the specific sensor on the FPGA. As such, the output of
the current sensor is connected FPGA\_ADC0, which is one of two ADC channels
available on the FPGA module.

\subsubsection{ADC}
The TeachEE ADC schematic sheet contains AD9288BSTZ dual channel 40MSPS ADC.
This chip provides two additional high-speed voltage channels on top of the ADC
channels provided directly on board the FPGA module. The data outputs for both
channels A and B are sent to the FPGA along with the sample clocks. This ADC was
selected for its FPGA friendly control interface. Since the control consists of
only a data bus and sample clock, it is trivial to write an FPGA module to
collect the data in comparison to other SPI and I2C based converters on the
market.

Bypass capacitors and ferrite beads are used to power the ADC for the cleanest
reference voltage possible. Additionally, the PCB contains non-populated
footprints for low pass filters, and filter bypasses to simplify testing and
debugging. For the digital side, all data bus and clocks have test points so a
multimeter or oscilloscope can be connected to debug the FPGA control signals.

\subsubsection{Analog Front End (AFE)}
The Analog Front End (AFE) sits in front of the inputs of the high-speed ADC. As
a result, the sub-circuit is duplicated twice in the top level sheet shown in Figure
\ref{fig:hw-top-level-sheet} to cover both channels. The AFE serves the following
purposes in the data acquisition hardware.

\begin{enumerate}
  \item Protect the ADC from over-voltage conditions. The front-end must clamp
    any voltages that would otherwise destroy the ADC.
  \item Convert the signal from single ended to differential. The ADC takes in a
    differential input, so the AFE must convert the input voltage observed at
    the connector to a differential signal of equivalent magnitude.
\end{enumerate}

This functionality is implemented in hardware using an AD8138 differential ADC
driver. This ``all-in-one'' component takes care of voltage clamping and
single to differential conversion. Moreover, it has sufficient analog bandwidth
to avoid reducing the effective bandwidth of the ADC. For debugging, the
sub-circuit contains 0-ohm bypass resistors that can be optionally soldered to
send the unclamped signal directly to the ADC.

\subsubsection{FPGA Module}
The FPGA module is implemented using a CMOD A7-35T Xilinx Spartan 7 Breakout
Board. the module is connected to the PCB via two 28-pin headers. Two pins are
used for power and ground while the rest are used for FPGA IO pins. The USB FIFO
and ADC control pins are connected to the FPGA IO. There also two special IO,
which can be used as ADCs. These two pins are used to implement the current and
voltage channels at 1MSPS. The current sensor output is wired to one of the
channels and a BNC connector is connected to the other through a low pass filter
for voltage sampling.

The FPGA module schematic had one mistake that was not discovered until the
physical PCBs had already been manufactured and delivered. In order to use the
parallel-bus interface of the FT232HQ to send data, the FPGA must lock to a 60
MHz output clock from the FT232HQ. During the layout, the FPGA pin wired to this
clock was swapped to a normal GPIO pin rather than an MRCC dedicated clock pin.
As a result, the FPGA place and route was failing during compilation as the FPGA
could not route this clock from a standard GPIO pin. This issue was resolved by
waiving the warnings and optimizing the SystemVerilog code to still pass timing
without a proper clock route for the 60 MHz signal. Further information on this
process is given in Section \ref{sec:fpga-impl} on the FPGA implementation.

\subsubsection{Connectors}
% Talk about LPFs and BNC connectors 
The connectors page contains all the external IO connections for TeachEE.
Specifically, the USB mini connector for data and three BNC connectors for scope
probes. The current sensor is exposed on its own separate pin header. The BNC
connectors are connected to the appropriate ADC input or AFE. Also, each
connectors has the appropriate low pass filters to satisfy nyquist theorem. Each
connector also includes all the standard capacitors and resistors needed to
connect an oscilloscope probe from the lab kit sold at the Queen's bookstore.
\subsection{FPGA} \label{sec:fpga-impl} % ETHAN
% Refer to system block diagram and go module by module.
% put in the top level module code and some critical state machines
% Talk about AXIS as a standard protocol. Also the extensible to ARM SOCs it
% provides

The FPGA datapath and implementation can be best summarized by examining the
interconnections between modules in the top-level SystemVerilog module.



\subsection{Software} % ERIC
% Basically everything from poster with more detail

\section{Testing, Evaluation \& Verification} \label{sec:testing}
The following tables outline the target hardware and software specifications of
this project.
\subsection{PCB Verification} % ETHAN
% Hardware tests
\subsection{FPGA Verification} % ETHAN
% Talk about FPGA automated sims, testbenches CI ETC
\subsection{Software Verification} % ERIC
% Rust CI, Testing with mock byte streams? 
\subsection{System Level Verification} % ETHAN / ERIC
% This subsection should include some notes on time spent in the lab running all
% three HW, FPGA and SW components integrated together. Also comparing our
% output against that of a real scope connected to the same signal

\section{Project Planning and Budgeting} % ERIC

\section{Stakeholder Needs} % John / Tim
%% Describe how you considered stakeholder needs in your design, and how factors
%% like safety, privacy, codes/standards, manufacturability, ethics and cost were
%% considered in your design.

% JOHN should use this section to discuss why our device is more suited to EE
% undergrads than what is currently on the market. Tim should provide some input
% on manufacturability and cost too.

\section{Compliance with System Requirements} % John
%% Compliance with specifications – Include your original specifications table from
%% the Blueprint and add extra column to it as shown below and report what you
%% obtained with your final design. If any of your specs were not achieved or fell
%% outside the tolerance values, explain why in this section. Your mark for this
%% section depends on how close you got to your specs

\section{Conclusions \& Recommendations} % Tim
%% Conclusions and recommendations – provide commentary on the main technical
%% lessons learned from this project; is there potential for lasting impact for
%% this project beyond this course? Can more research be done in an area? Should
%% someone tackle this problem again but using a different approach? Is there
%% potential for commercialization? If this were to be scaled out to a commercial
%% version how would it be manufactured and what would the cost be? Speculate about
%% the market size for such a product. If the product makes it to the market, what
%% are the potential positive and adverse societal and cultural impacts of the
%% product? How can the adverse impacts be mitigated?

\section{Overall Team Effort}
The following table quantifies the percentage effort each team member expended
on all aspects of the project.

\begin{table}[H]
  \caption{Team Effort Table}
  \centering
  \begin{tabularx}{10cm}{l|l}
    \textbf{Name} & \textbf{Effort Expended \%}\\
    \hline
    Ethan Peterson & TODO \\
    \hline
    Eric Yang & TODO \\
    \hline
    Timothy Morland & TODO \\
    \hline
    John Giorshev & TODO \\
  \end{tabularx} 
\end{table}
\newpage

\bibliographystyle{IEEEtran}
\bibliography{report}

\newpage
\setstretch{1}

\pagestyle{empty}

    \begin{appendices}
        \section{PCB System Block Diagram}
        \label{appendix:block-diagram}
        \begin{figure}[H]
            \centering
            % \includegraphics[width=16cm]{../../misc/TeachEE-System-Diagram.drawio.png}
            \caption{TeachEE PCB System Block Diagram}
            \label{fig:pcb-block-diagram}
        \end{figure}
        \begin{landscape}
        \section{Schematics}
        \label{appendix:schematic}
    \centering
    \includegraphics[height=15cm]{schematics/schematic1-1.png}
        \end{landscape}
    \includepdf[pages=-,landscape=true]{schematics/schematic2-8.pdf}

        \begin{landscape}
        \section{PCB Layout}
        \label{appendix:layout}
    \includegraphics[height=15cm]{schematics/Layout.png}
        \end{landscape}

        \section{PCB Bill of Materials}
        \label{appendix:bom}
    \begin{table}[H]
    \begin{tabular}{p{1.7in}|p{0.3in}|p{0.8in}|p{0.7in}|p{0.7in}|p{1in}}
        \textbf{Designator} & \textbf{QTY} & \textbf{Description} & \textbf{Comment} & \textbf{Footprint} & \textbf{Part Number} \\ \hline 
        C1, C2, C37, C48 & 4 & CAP CER 10UF 10V X5R 0402 & 10 µF & CAP 0402\_1005 & CL05A106MP8NUB8 \\ \hline
        C3 & 1 & CAP CER 1UF 10V X7S 0402 & 1 µF & CAP 0402\_1005 & GRM155C71A105KE11D \\ \hline
        C4, C11, C16, C17, C18, C19, C20, C21, C22, C24, C26, C29, C30, C38, C39, C40, C41, C42, C43, C44, C45, C46, C47, C50\_AFE\_CHANNEL\_A, C50\_AFE\_CHANNEL\_B, C52\_AFE\_CHANNEL\_A, C52\_AFE\_CHANNEL\_B, C53\_AFE\_CHANNEL\_A, C53\_AFE\_CHANNEL\_B & 29 & CAP CER 0.1UF 50V X5R 0402 & 0.1 µF & CAP 0402\_1005 & CGA2B3X5R1H104M050BB \\ \hline
        C5, C12, C32, C34 & 4 & CAP CER 0.1UF 10V X7R 0402 & 0.1µF & CAP 0402\_1005 & 0402ZC104KAT2A \\ \hline
        C6 & 1 & CAP CER 10UF 16V X5R 1206 & 10 µF & CAP 1206\_3216 - 0.8MM & EMK316BJ106MD-T \\ \hline
        C7 & 1 & CAP CER 10UF 35V X6S 0805 & 10 µF & CAP 0805\_2012 & GRM21BC8YA106KE11L \\ \hline
        C8, C15, C23, C25 & 4 & CAP CER 4.7UF 16V X5R 0402 & 4.7 µF & CAP 0402\_1005 & CL05A475MO5NUNC \\ \hline
        C9, C27, C28 & 3 & CAP CER 20PF 25V NP0 0402 & 20pF & CAP 0402\_1005 & 04023U200JAT2A \\ \hline
    \end{tabular}
    \end{table}

    \newpage

    \begin{table}[H]
    \begin{tabular}{p{1.7in}|p{0.3in}|p{0.8in}|p{0.7in}|p{0.7in}|p{1in}}
        \textbf{Designator} & \textbf{QTY} & \textbf{Description} & \textbf{Comment} & \textbf{Footprint} & \textbf{Part Number} \\ \hline 
        C10 & 1 & CAP MLCC 0.01UF 100V X7R 0402 & 10000 pF & CAP 0402\_1005 & HMK105B7103KVHFE \\ \hline
        C13, C14 & 2 & CAP CER 36PF 100V NP0 0603 & 36pF & CAP 0603\_1608 & 06031A360JAT2A \\ \hline
        C31 & 1 & CAP CER 1UF 16V X7R 0603 & 1 µF & CAP 0603\_1608 & C0603C105K4RACAUTO \\ \hline
        C33, C35 & 2 & CAP CER 0603 4.7NF 16V X7R 10\% & 4700 pF & CAP 0603\_1608 & C0603C472K4RECAUTO \\ \hline
        C51\_AFE\_CHANNEL\_A, C51\_AFE\_CHANNEL\_B & 2 & CAP CER 15PF 25V C0G/NP0 0603 & 15 pF & CAP 0603\_1608 & C0603C150J3GACAUTO \\ \hline
        D1, D3 & 2 & LED SMD & Blue & LED 1206\_3216 BLUE & APTL3216QBC/D-01 \\ \hline
        D2, D5 & 2 & LED SMD & Green & LED 1206\_3216 GREEN & APTL3216ZGCK-01 \\ \hline
        FB1, FB2, FB3 & 3 & FERRITE BEAD 33 OHM 0201 1LN & 33 Ohms @ 100 MHz & FER 0201\_0603 & MMZ0603F330CT000 \\ \hline
        J1, J3, J4 & 3 & Jack BNC Connector, 1 Position, Height 16.26 mm, Tail Length 6.35 mm, -55 to 85 degC, RoHS, Tube & 5227699-2 & & 5227699-2 \\ \hline
        J2 & 1 & CONN RCPT MINI USB B 5POS SMD RA & & & 10033526-N3212LF \\ \hline
    \end{tabular}
    \end{table}
    \newpage
    \begin{table}[H]
        \begin{tabular}{p{1.7in}|p{0.3in}|p{0.8in}|p{0.7in}|p{0.7in}|p{1in}}
        \textbf{Designator} & \textbf{QTY} & \textbf{Description} & \textbf{Comment} & \textbf{Footprint} & \textbf{Part Number} \\ \hline 
        R1, R2 & 2 & RES SMD 10 OHM 0.1\% 1/10W 0603 & 10 Ohms & RES 0603\_1608 & CRT0603-BY-10R0ELF \\ \hline
        R3, R6, R13 & 3 & RES 220 OHM 1\% 1/8W 0402 & 220 Ohms & RES 0402\_1005 & CRGP0402F220R \\ \hline
        R4 & 1 & RES SMD 330 OHM 5\% 1/16W 0402 & 330 Ohms & RES 0402\_1005 & AC0402JR-07330RL \\ \hline
        R5 & 1 & 1206 40 AMP JUMPER & 0 Ohms & RES 1206\_3216 & JR1206X40E \\ \hline
        R7 & 1 & RES SMD 12K OHM 0.1\% 1/16W 0402 & 12 kOhms & RES 0402\_1005 & CPF0402B12KE1 \\ \hline
        R8, R9, R10, R11, R17, R18, R19, R21, R32, R33, R34 & 11 & RES 10K OHM 0.1\% 1/10W 0402 & 10 kOhms & RES 0402\_1005 & RP73PF1E10KBTD \\ \hline
        R12 & 1 & RES 1.8K OHM 1\% 1/16W 0402 & 1.8 kOhms & RES 0402\_1005 & RC0402FR-071K8L \\ \hline
        R14 & 1 & RES SMD 100K OHM 0.1\% 1/16W 0402 & 100 kOhms & RES 0402\_1005 & CPF0402B100KE \\ \hline
        R15, R46 & 2 & RES SMD 0 OHM JUMPER 1/2W 0603 & 0 Ohms & RES 0603\_1608 & 5110 \\ \hline
        R22 & 1 & RES SMD 75 OHM 1\% 1/10W 0603 & 75 Ohms & RES 0603\_1608 & AC0603FR-0775RL \\ \hline
        R26 & 1 & RES 24 OHM 1\% 1/16W 0402 & 24 Ohms & RES 0402\_1005 & RC0402FR-0724RL \\ \hline
        \end{tabular}
    \end{table}
    \newpage
    \begin{table}[H]
        \begin{tabular}{p{1.7in}|p{0.3in}|p{0.8in}|p{0.7in}|p{0.7in}|p{1in}}
        \textbf{Designator} & \textbf{QTY} & \textbf{Description} & \textbf{Comment} & \textbf{Footprint} & \textbf{Part Number} \\ \hline 
        R27, R28, R29 & 3 & RES 1M OHM 1\% 1/16W 0402 & 1 MOhms & RES 0402\_1005 & RMCF0402FT1M00 \\ \hline
        R36\_AFE\_CHANNEL\_A, R36\_AFE\_CHANNEL\_B & 2 & RES SMD 523 OHM 0.1\% 1/16W 0402 & 523 Ohms & RES 0402\_1005 & ERA-2ARB5230X \\ \hline
        R37\_AFE\_CHANNEL\_A, R37\_AFE\_CHANNEL\_B, R40\_AFE\_CHANNEL\_A, R40\_AFE\_CHANNEL\_B, R41\_AFE\_CHANNEL\_A, R41\_AFE\_CHANNEL\_B & 6 & RES SMD 500 OHM 0.05\% 1/10W 0603 & 500 Ohms & RES 0603\_1608 & TNPU0603500RAZEN00 \\ \hline
        R39\_AFE\_CHANNEL\_A, R39\_AFE\_CHANNEL\_B, R43\_AFE\_CHANNEL\_A, R43\_AFE\_CHANNEL\_B & 4 & RES 50 OHM 5\% 1/8W 0603 & 50 Ohms & RES 0603\_1608 & CH0603-50RJNTA \\ \hline
        R42\_AFE\_CHANNEL\_A, R42\_AFE\_CHANNEL\_B & 2 & RES SMD 4.02K OHM 1\% 1/10W 0603 & 4.02 kOhms & RES 0603\_1608 & AC0603FR-074K02L \\ \hline
        R44\_AFE\_CHANNEL\_A, R44\_AFE\_CHANNEL\_B & 2 & RES SMD 1K OHM 1\% 1/10W 0603 & 1 kOhms & RES 0603\_1608 & AA0603FR-071KL \\ \hline
        R45\_AFE\_CHANNEL\_A, R45\_AFE\_CHANNEL\_B & 2 & RES 25 OHM 0.1\% 1/20W 0402 & 25 Ohms & RES 0402\_1005 & FC0402E25R0BST0 \\ \hline
        S1 & 1 & SWITCH PUSH SPST-NO 0.4VA 28V & &  & AB11AH-HA \\ \hline
        TP1, TP2, TP3, TP4, TP5, TP9, TP24, TP25, TP26, TP45, TP46, TP47, TP48, TP50, TP51, TP52, TP53, TP54, TP55, TP56 & 20 & Test Point, 1 Position SMD, RoHS, Tape and Reel & 5019 & KSTN5019 & 5019 \\ \hline
        U1 & 1 & IC HS USB TO UART/FIFO 48QFN & FT232 & QFN-48 & FT232HQ-REEL \\ \hline
        \end{tabular}
    \end{table}
    \newpage
    \begin{table}[H]
        \begin{tabular}{p{1.7in}|p{0.3in}|p{0.8in}|p{0.7in}|p{0.7in}|p{1in}}
        \textbf{Designator} & \textbf{QTY} & \textbf{Description} & \textbf{Comment} & \textbf{Footprint} & \textbf{Part Number} \\ \hline 
        U2 & 1 & Dual 8-Bit AD Converter with Parallel Interface, 40MSPS, -40 to +85 degC, ST-48, Pb-Free, Tray &  & ST-48M & AD9288BSTZ-40 \\ \hline
        U3\_AFE\_CHANNEL\_A, U3\_AFE\_CHANNEL\_B & 2 & IC ADC DRIVER 8SOIC & AD8138 & R-8-IPC\_A & AD8138ARZ-R7 \\ \hline
        U5 & 1 & Fixed Low Drop Positive Voltage Regulator, 3.3V, 3-Pin TO-220 & LD1117 & TO220 & LD1117V33C \\ \hline
        U6 & 1 & 2K, 128x16-bit, 2.5V Microwire Serial EEPROM, 8-Pin SOIC 150mil, Commercial Temperature, Tape and Reel & 93LC56 & SOIC8 & 93LC56BT/SN \\ \hline
        U7 & 1 & CURRENT SENSOR & ACS720 &  & ACS720KLATR-15AB-T \\ \hline
        Y1 & 1 & CRYSTAL 12.0000MHZ 20PF SMD & 12 MHz & ABLS & ABLS-12.000MHZ-20-B-3-H-T \\ \hline
        \end{tabular}
    \end{table}

    \section{TeachEE SystemVerilog RTL Code} \label{appendix:rtl-code}
    This appendix contains all the SystemVerilog RTL files in the TeachEE
    project. Each subsection corresponds to the files in a particular folder of
    the GitHub repository.

    \subsection{AXIS} \label{appendix:axis}
    \subsubsection{axis\_adapter\_wrapper.sv}
    \inputminted{systemverilog}{../../rtl/axis/axis_adapter_wrapper.sv}
    \subsubsection{axis\_async\_fifo\_wrapper.sv}
    \inputminted{systemverilog}{../../rtl/axis/axis_async_fifo_wrapper.sv}
    \subsubsection{axis\_interface.sv}
    \inputminted{systemverilog}{../../rtl/axis/axis_interface.sv}
    
    \end{appendices}
\end{document}
